\noindent
\emph{The reviewers felt the reviews were shallow, but the scientists appreciated them.}

Overall, all of the reviewers were satisfied that they were able to provide a review with some utility, but they were also frustrated because their review was constrained by lack of domain knowledge, lack of communication with the author – and hence lack of context. Therefore, they felt that their reviews were ``limited'' or ``shallow''. A number of reviewers noted that they were unable to run the code, which most reviewers (but not all) would normally do. As a result, the review comments were at the algorithmic, syntactic, and stylistic levels; they were not able to comment on structure (except at low level), organization or architecture.
In contrast, all of the code authors found the comments useful: typical responses were, ``detailed'', ``knowledgeable reviewer'', ``presentation clear'', ``gave alternative suggestion'', and ``useful feedback''. Feedback the scientists found particularly valuable focused on:
\begin{itemize}
\item usability (e.g., how easily someone can ``enter the structure'' of the code)
\item ease of re-use, readability, and density
\item code structure and solution structure (``The way they organise a solution to a problem'')
\item feedback on the organisation of README files
\item performance and opttimization
\item unit testing (or its absence)
\end{itemize}
From this, we may conclude that there is low-hanging fruit in scientific code review: things that do not seem challenging to software developers are nevertheless seen to be useful by scientists who are striving to introduce a culture of code quality.


\begin{quotation}
  \noindent
  I also must confess to a strong bias against the fashion for reusable code.
  To me, ``re-editable code'' is much, much better than an untouchable black box or toolkit.
  I could go on and on about this.
  If you're totally convinced that reusable code is wonderful,
  I probably won't be able to sway you anyway,
  but you'll never convince me that reusable code isn't mostly a menace.
  \\
  -- Donald E. Knuth, interviewed by Andrew Binstock, 2008-04-25
\end{quotation}
